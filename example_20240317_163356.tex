\documentclass[12pt]{article}
\usepackage{graphicx} 
\usepackage{biblatex} 
\usepackage{CJKutf8}
\usepackage[T1]{fontenc}
\usepackage{color}
\usepackage{xcolor}
\usepackage{tikz}
\usepackage{float}
\usepackage[calc]{datetime2}
\usepackage{fancyhdr}
\usepackage[printwatermark]{xwatermark}
\usetikzlibrary{calc}
\usepackage[top=1in, bottom=1in, left=1in, right=1in]{geometry}
\usepackage{moresize}
\newwatermark[pages=2-30,color=blue!15,angle=45,scale=3,xpos=0,ypos=0]{Rapidamic Lab}
\pagestyle{fancy}
\fancyhf{}
\definecolor{darkblue}{RGB}{0,90,146} 
\fancyhead[L]{\textnormal{\textcolor{black}{素研实验室}}} 
\fancyhead[R]{\includegraphics[width=0.7cm]{logo.jpg}}
\usepackage{tcolorbox}%
\newcommand{\MyTitle}[4][]{%
\begin{tcolorbox}[width=\textwidth, arc=0mm, auto outer arc,
   boxrule=-1pt, toprule=5pt, bottomrule=5pt,
   colframe=suyanblue, colback=white,
   top=0.5cm, bottom=0.1cm]
   \end{tcolorbox}
\end{tcolorbox}%
}%
\newcommand{\insertname}{
    Jake Li
}
\newcommand{\customdate}{
    \the\year年\hspace{0.5em}%
    \ifnum\month<10 0\fi\the\month月\hspace{0.5em}%
    \ifnum\day<10 0\fi\the\day日%
}
\renewcommand{\headrulewidth}{3pt} 
\renewcommand{\headrule}{\hbox to\headwidth{\color{darkblue}\leaders\hrule height \headrulewidth\hfill}} 
\begin{document}
\thispagestyle{empty} 
\begin{CJK*}{UTF8}{gbsn}
\definecolor{darkblue}{RGB}{0, 90, 146}

\noindent\hfill\begin{minipage}{2.0cm} % set logo on the title page
   \includegraphics[width=\textwidth]{logo_.jpg}\\
\end{minipage}
\vspace{-3.7cm}

\noindent\textcolor{darkblue}{\rule{\textwidth}{5pt}} % Increase the thickness of the top line

\vspace{10pt} % Increase vertical space for visual separation
{\noindent\Large\bfseries\hspace*{0.5cm}ChatIvy专专 业 匹 配 及 探 索 报 告}\\[12pt] % Increase font size and space after the title
{\noindent\large\bfseries\hspace*{0.5cm}报告收件人: \insertname}\\[10pt] % Increase font size and space after the recipient line
{\noindent\large\bfseries\hspace*{0.5cm}测 试 日 期 :\customdate}
\vspace{10pt} % Increase vertical space before the bottom line

\noindent\textcolor{darkblue}{\rule{\textwidth}{5pt}} % Increase the thickness of the bottom line



   \begin{center}
      \centering
      \includegraphics[width=20cm]{/home/hello/VScode/sigma.jpg}
   \end{center}
   \vspace{15mm}
   \begin{tikzpicture}[remember picture, overlay]
      % Define colors
      \definecolor{darkblue}{RGB}{0,90,146} % Color for the stripe
      \definecolor{darkerblue}{RGB}{0,35,102} % Darker color for the frame
   
      % Adjusted frame and stripe positions to move up by 2cm
      \fill[darkerblue] ($(current page.south west)+(1.9cm,1.4cm + 2cm)$) rectangle ($(current page.south east)+(-1.9cm,2.6cm + 2cm)$);
      \fill[darkblue] ($(current page.south west)+(2.0cm,1.5cm + 2cm)$) rectangle ($(current page.south east)+(-2.0cm,2.5cm + 2cm)$);
   
      % Text on the stripe with increased font size, also moved up by 2cm
      \node at ($(current page.south)+(0,2cm + 2cm)$) [white, font=\large] {基于您输入的信息生成的可视化词云};
   \end{tikzpicture}


   %generated_summary
   \newpage
   \fancyfoot[C]{\thepage}
   \vspace*{1cm}
   亲爱的Jake Li,

我很高兴有机会了解你的想法和计划。在你的问卷中,我看到了一位开放、勇于尝试新事物的年轻人。你对未知充满好奇,愿意探索新领域,这是一种非常宝贵的品质。

然而,我也注意到你目前还没有对大学专业做出明确的选择。这并不是问题,实际上,许多人在进入大学之前都不确定他们想要追求什么专业。重要的是保持开放和好奇心,并尽可能多地了解各种可能性。

在选择专业时,你似乎并不过分关注父母能提供的资源。这表明你有独立思考和决策的能力,并且愿意根据自己真正感兴趣和擅长的事情来做出选择。

至于本科毕业后的计划,你目前还没有确定。这也完全正常。大学生活将为你提供许多机会去探索自己对未来可能感兴趣的职业道路。

总之,Jake, 请继续保持开放和好奇心,并积极寻找适合自己的道路。记住,在这个过程中没有正确或错误的选择,只有最适合你的选择。

祝好运,

长者
   \begin{flushleft}
   \textit{\textnormal{ChatIvy.}}
   \end{flushleft}
   \newpage
   \hspace{0pt}
   \vspace{0cm}
   
   \begin{flushleft}
   本次的报告主要由以下部分组成:
   \begin{itemize} 
       \item \underline{最佳匹配专业}:根据问卷信息以及权重,为用户推荐最适合的3个专业以及推荐理由与匹配度。报告同时包含3个专业的潜在专业缺点困境,为用户进一步的排除掉不喜欢的专业。
       \item \underline{专业对应大学推荐}:根据问卷信息,为用户推荐他/他希望就读国家的2个推荐专业最具代表性的大学以供用户参考。
       \item \underline{推荐专业对应的基础和进阶课程}:列出为用户推荐的3个最匹配的专业在大学中的基础与进阶课程,加深用户对于未来专业会学习到的内容的了解,帮助用户筛选出最终专业。
       \item \underline{推荐专业历史发展}:介绍为用户推荐的三个专业过去50年发展历史中的重要转折点及其影响意义,扩展了用户对于匹配专业的认识。
       \item \underline{推荐专业前沿领域}:介绍为用户推荐的三个专业在工业界和学术界的较前沿领域,为用户未来专业领域选择提供帮助。
       \item \underline{学科兴趣和能力的多维度评估可视化}:根据问卷信息及用户对五大学科细分领域的感兴趣程度,评估用户在推荐的三个最佳匹配专业和学科四大类里基于五大不同维度的匹配度,并进行可视化分析。
       \item \underline{基于推荐专业的高中活动规划}:根据问卷信息,为用户规划基于3个不同专业发展方向的高中活动,包括一周以内的活动,一个月以内的活动,一年以内的活动以及背景提升规划的活动以供用户参考。
       \item \underline{其他可选择专业}:列出7个符合用户要求,但匹配度非最高的专业,以供用户额外参考。
   \end{itemize}
   \end{flushleft}
   \hspace{0pt}
   \vfill
   


   % \bigskip
   % \bigskip
   % \vfill
   % \input{figure_tex/percentile}
   % \vfill

%generated_major_prompt_two
   \newpage
   \subsection*{最佳匹配专业}
   根据您提供的信息,以下是我们为同学推荐的最适合他的3个专业:
      
   \textbf{1.} 根据Jake Li的问卷信息,我为他推荐以下10个专业:

1. 计算机科学:Jake对于新事物的接受度很高,这对于计算机科学这个不断更新和发展的领域来说是非常重要的。同时,他并没有对大学专业有明确的选择,而计算机科学作为一个前景广阔、就业率高的专业,可以给他提供更多可能性。

2. 商务管理:Jake在问卷中表示父母可提供资源在他选择专业中起到一定作用。商务管理需要良好的社交网络和资源支持,在这方面父母可以给予很大帮助。

3. 金融:金融是一个需要不断尝试新事物、探索新方法的领域。Jake愿意尝试新事物,这将有利于他在金融领域取得成功。

4. 市场营销:市场营销需要创新思维和敏锐洞察力。Jake愿意尝试新事物表明他具备创新精神,而且市场营销也是一个可以充分利用父母资源的专业。

5. 数据科学:数据科学是一个充满挑战和未知性的领域。Jake愿意尝试新事物说明他具备应对挑战的勇气和决心。

6. 人力资源管理:人力资源管理需要良好的社交能力和人际关系处理能力。Jake愿意尝试新事物,这说明他有可能具备这些能力。

7. 国际商务:国际商务需要广阔的视野和敏锐的洞察力。Jake愿意尝试新事物,这表明他有可能具备这些素质。

8. 工程管理:工程管理是一个需要不断学习新知识、掌握新技术的领域。Jake愿意尝试新事物,这将有利于他在工程管理领域取得成功。

9. 电子商务:电子商务是一个充满变化和创新的领域。Jake愿意尝试新事物,这将有利于他在电子商务领域取得成功。

10. 环境科学:环境科学是一个需要大量实践和探索的领域。Jake愿意尝试新事物,这将有利于他在环境科学领域取得成功。

以上推荐基于Jake对各因素权重的判断以及我多年留学咨询经验所做出。每个专业都与Jake所给出因素权重相匹配,并且考虑到了未来就业前景和发展潜力。


%generated_major_prompt_three
   \newpage
   \subsection*{潜在的专业困境}\textbf{2.} 由于提供的信息中,本科毕业去向为待定,且没有列出其他的考虑或要求,因此我们需要根据前面的信息来进行第二步的筛选。

1. 专业一:计算机科学与技术(Computer Science and Technology)

理由:首先,该专业是STEM专业之一,符合学生对STEM专业的偏好。其次,在现代社会中,计算机科学与技术是一个非常重要且广泛应用的领域。无论是在工作还是生活中,都离不开计算机科学与技术。此外,在全球范围内,计算机科学与技术也是一个非常热门且有很高就业率和薪资水平的专业。因此,选择这个专业将为未来打开更多可能性和机会。

匹配度:90

2. 专业二:电子信息工程(Electronic Information Engineering)

理由:电子信息工程也是STEM领域内的一个重要分支。这个专业涵盖了电子、通信、控制等多个方面的知识,并且在实际应用中有着广泛的应用前景。例如,在智能设备、网络通信、自动化控制等领域都有着广泛应用。而且随着科技发展和社会进步,电子信息工程的应用领域还在不断扩大。因此,选择这个专业将有助于学生在未来找到更多的就业机会。

匹配度:85

3. 专业三:机械工程(Mechanical Engineering)

理由:机械工程是一门涵盖了力学、材料科学、设计制造等多个方面的综合性专业。这个专业不仅可以提供丰富的理论知识,还可以提供实践操作的机会,使得学生能够更好地理解和掌握相关知识。此外,随着科技发展和社会进步,机械工程也在不断发展和创新,为未来提供了更多可能性。因此,选择这个专业将有助于学生在未来找到更多的就业机会。

匹配度:80



%generated_potential_major_

   \textbf{3.}
   1. 计算机科学与技术:
   - 学习难度:计算机科学与技术涉及到大量的理论知识和编程技能,需要花费大量的时间和精力去学习。例如,数据结构、算法、操作系统等都是非常复杂的课程,需要深入理解并掌握。
   - 实践机会:虽然计算机科学与技术有很多实践项目,但是由于其复杂性,很多时候学生可能无法在短时间内将理论知识转化为实际应用。例如,在开发一个软件时,可能需要花费数周甚至数月的时间。
   - 职业前景:虽然计算机科学与技术的就业前景广阔,但是由于竞争激烈,没有足够经验或者技能的毕业生可能会面临就业压力。例如,在求职过程中可能会遇到大量具有丰富经验和高级证书的竞争者。
   - 竞争压力:由于计算机科学与技术是一个热门专业,因此竞争压力非常大。例如,在求职过程中可能会遇到大量具有丰富经验和高级证书的竞争者。
   - 专业要求:计算机科学与技术的课程要求严格,需要学生在短时间内掌握大量的知识和技能。例如,数据结构、算法、操作系统等都是必修课程。
   - 工作压力:计算机科学与技术的工作压力大,常常需要长时间工作,并且需要不断更新知识和技能以应对快速变化的行业环境。例如,在软件开发过程中可能会遇到紧急的bug需要修复,或者需要在短时间内完成项目。
   - 对经济环境或政策的依赖:计算机科学与技术并不直接受到经济环境或政策变化的影响。
   - 快速变化的行业环境:计算机科学与技术是一个快速发展和变化的行业,需要持续学习和更新知识和技能。例如,新的编程语言、框架和工具不断出现,如果不能跟上这些变化可能会影响到工作。

2. 电子信息工程:
   - 学习难度:电子信息工程涉及到大量复杂的理论知识和实验操作,如电路分析、信号处理等。这些都需要深入理解并掌握。
   - 实践机会:电子信息工程的实践机会相对较少,很多时候学生可能只能在实验室中进行模拟操作。例如,在学习电路设计时,可能只能通过软件进行模拟,而无法在真实环境中进行操作。
   - 职业前景:电子信息工程的就业前景相对较窄,主要集中在硬件设计、通信设备制造等领域。例如,在求职过程中可能会发现大部分职位都要求有丰富的硬件设计经验。
   - 竞争压力:由于电子信息工程是一个专业性强的专业,因此竞争压力相对较小。但是,在一些热门领域如芯片设计等,竞争压力仍然存在。
   - 专业要求:电子信息工程的课程要求严格,需要学生掌握大量的理论知识和实验技能。例如,数字信号处理、微波技术等都是必修课程。
   - 工作压力:电子信息工程的工作压力相对较大,常常需要长时间进行精细的设计和调试。例如,在芯片设计过程中可能需要花费数周甚至数月的时间来优化和调试。
   - 对经济环境或政策的依赖:电子信息工程受到经济环境和政策变化的影响较大。例如,全球芯片短缺的问题就直接影响到了这个行业的发展。
   - 快速变化的行业环境:电子信息工程是一个快速发展和变化的行业,需要持续学习和更新知识和技能。例如,新的设计方法、工具和材料不断出现,如果不能跟上这些变化可能会影响到工作。

3. 机械工程:
   - 学习难度:机械工程涉及到大量复杂的理论知识和实验操作,如力学、热力学、流体力学等。这些都需要深入理解并掌握。
   - 实践机会:机械工程有一定的实践机会,但是由于设备成本高昂,很多时候学生可能只能在实验室中进行模拟操作。例如,在学习车辆设计时,可能只能通过软件进行模拟,而无法在真实环境中进行操作。
   - 职业前景:机械工程的就业前景相对较窄,主要集中在汽车制造、航空航天等领域。例如,在求职过程中可能会发现大部分职位都要求有丰富的设计经验。
   - 竞争压力:由于机械工程是一个专业性强的专业,因此竞争压力相对较小。但是,在一些热门领域如汽车设计等,竞争压力仍然存在。
   - 专业要求:机械工程的课程要求严格,需要学生掌握大量的理论知识和实验技能。例如,材料力学、机械设计等都是必修课程。
   - 工作压力:机械工程的工作压力相对较大,常常需要长时间进行精细的设计和调试。例如,在产品设计过程中可能需要花费数周甚至数月的时间来优化和调试。
   - 对经济环境或政策的依赖:机械工程受到经济环境和政策变化的影响较大。例如,全球汽车行业转向电动化就直接影响到了这个行业的发展。
   - 快速变化的行业环境:机械工程也面临着快速发展和变化的行业环境,需要持续学习和更新知识和技能。例如,新型材料、制造方法不断出现,如果不能跟上这些变化可能会影响到工作。

%major_list_
%generated_Correspondence_college_recommendations_
   \newpage
   \subsection*{专业对应大学推荐}
   美国:

1. 计算机科学与技术:斯坦福大学和麻省理工学院是美国在计算机科学与技术领域的两所顶尖大学。斯坦福大学位于硅谷,周边有许多世界顶级的科技公司,如Google、Facebook等,为学生提供了丰富的实习和就业机会。同时,该校在计算机科学领域有着强大的教师团队和研究实力。麻省理工学院也是全球计算机科学领域的重要研究中心之一,其课程设置兼具理论性和实践性,能够让你在深入理解计算机科学原理的同时,掌握最前沿的技术。

2. 电子信息工程:加州伯克利大学和伊利诺伊大学香槟分校都是电子信息工程领域非常出色的选择。加州伯克利大学电子工程与计算机科系被公认为全球最好之一,在半导体、集成电路设计等方向有着深厚积累。而伊利诺伊大学香槟分校则以其在微电子、信号处理等方向的研究实力闻名。

3. 机械工程:麻省理工学院和加州理工学院是美国在机械工程领域的两所顶尖大学。麻省理工学院的机械工程专业以其全面、创新的课程设置和强大的教师团队而闻名,同时,该校还有许多与行业巨头合作的项目,为学生提供了丰富的实践机会。加州理工学院则以其在航空航天、能源等方向上的研究实力著称。

英国:

1. 计算机科学与技术:剑桥大学和牛津大学是英国在计算机科学与技术领域最好的两所大学。剑桥大学计算机科系有着世界一流的教师团队和研究设施,同时,该校也非常注重实践教育,有许多企业合作项目供你选择。牛津大学则以其深厚的历史积淀和卓越的研究成果在全球享有盛誉。

2. 电子信息工程:伦敦帝国理工和南安普顿大学都是电子信息工程领域非常出色的选择。伦敦帝国理工的电子信息工程专业以其全面、创新的课程设置和强大的教师团队而闻名,同时,该校还有许多与行业巨头合作的项目,为学生提供了丰富的实践机会。南安普顿大学则以其在光电子、通信等方向上的研究实力著称。

3. 机械工程:剑桥大学和伦敦帝国理工是英国在机械工程领域最好的两所大学。剑桥大学机械工程专业以其全面、创新的课程设置和强大的教师团队而闻名,同时,该校还有许多与行业巨头合作的项目,为学生提供了丰富的实践机会。伦敦帝国理工则以其在航空航天、能源等方向上的研究实力著称。
   1. 计算机科学与技术
2. 电子信息工程
3. 机械工程


%generated_Correspondence_Courses_
   \newpage
   \subsection*{推荐专业对应的基础和进阶课程}
   1. 计算机科学与技术 (Computer Science and Technology)
   基础课程:
   - 程序设计基础 (Fundamentals of Programming)
   - 数据结构 (Data Structures)
   - 计算机组成原理 (Principles of Computer Organization)
   - 操作系统 (Operating Systems)
   - 计算机网络 (Computer Networks)
   - 数据库系统概论 (Introduction to Database Systems)
   - 软件工程基础 (Fundamentals of Software Engineering)

   进阶课程:
   - 人工智能原理与应用 (Principles and Applications of Artificial Intelligence)
   - 云计算与大数据处理技术(Cloud Computing and Big Data Processing Technology)
   - 信息安全基础(Information Security Fundamentals)

2. 电子信息工程(Electronic Information Engineering) 
    基础课程:
    - 高等数学(Advanced Mathematics) 
    - 大学物理(College Physics) 
    - 数字逻辑(Digital Logic) 
    - 电路分析基础(Fundamentals of Circuit Analysis) 
    - 模拟电子技术(Analog Electronics Technology) 
    - 数字电子技术(Digital Electronics Technology) 
    - 微型计算机原理(Principles of Microcomputer)

     进阶课程:
     - 单片微机原理与接口技术(Principle and Interface Technology of Single Chip Microcomputer)  
     - 现代通信原理(Modern Communication Principles)  
     – 高频电子线路(High Frequency Electronic Circuits)

3. 机械工程 (Mechanical Engineering)
    基础课程:
    - 工程图学 (Engineering Graphics)
    - 工程力学 (Engineering Mechanics)
    - 材料力学 (Material Mechanics)
    - 热力学基础 (Fundamentals of Thermodynamics)
    - 机械设计基础 (Fundamentals of Mechanical Design)
    - 制造工艺学基础 (Fundamentals of Manufacturing Technology)
    - 电气与电子技术基础(Fundamentals of Electrical and Electronic Technology)

     进阶课程:
     - 数控技术与编程(Numerical Control Technology and Programming) 
     – 机器人技术(Robotics Technology) 
     – 先进制造技术(Advanced Manufacturing Technology)
   % \input{figure_tex/course_table}
   
%generated_Major_development_history_
   \newpage
   \subsection*{推荐专业发展历史}
   以下信息介绍了为推荐的3个专业在近50年的重要转折点及其影响:
   1. 计算机科学与技术
   - 1970年,个人计算机的诞生:在这一年,世界上第一台个人计算机Altair 8800面世。这标志着计算机从大型机时代进入了个人计算机时代,极大地推动了计算机科学与技术的发展。
   - 1981年,IBM PC的发布:IBM发布了其第一台个人电脑(PC),并且开放硬件架构,使得其他厂商可以复制和生产兼容PC。这使得个人电脑开始普及,并推动了软件产业的发展。
   - 1989年,万维网(WWW)的诞生:提姆·伯纳斯-李发明了万维网,并在1993年将其开源。这使得互联网开始普及,并催生出了许多新的应用和服务。
   - 1995年,Java语言的发布:Sun公司发布了Java编程语言。Java语言简单、安全、跨平台,在网络编程、企业级应用等领域得到广泛应用。
   - 2007年,iPhone和Android系统的发布:苹果公司发布iPhone手机和iOS操作系统;Google公司发布Android操作系统。这标志着移动互联网时代的来临。

2. 电子信息工程
   - 1971年,微处理器的诞生:Intel公司发布了世界上第一款商用微处理器Intel 4004。这使得电子设备可以更小、更便宜、更强大。
   - 1977年,GPS系统的建立:美国开始建立全球定位系统(GPS)。这使得导航、定位等应用成为可能。
   - 1980年,数字信号处理器(DSP)的诞生:贝尔实验室发明了世界上第一款DSP芯片TMS32010。这推动了语音、图像等数字信号处理技术的发展。
   - 1991年,2G移动通信技术的商用:北欧移动电话(NMT)在芬兰开启了世界上第一个2G网络。这标志着移动通信从模拟时代进入了数字时代。
   - 2009年,4G移动通信技术的商用:瑞典和挪威首次开启4G网络。这使得高速移动互联网成为可能。

3. 机械工程
   - 1970年,数控机床的普及:数控机床开始在工业生产中广泛应用。这提高了生产效率和产品质量。
   - 1980年,CAD/CAM技术的发展:计算机辅助设计和制造(CAD/CAM)技术开始广泛应用。这使得产品设计和制造更加精确和高效。
   - 1990年,机器人技术的发展:工业机器人开始在汽车、电子等行业广泛应用。这进一步提高了生产效率。
   - 2000年,3D打印技术的发展:3D打印技术开始在产品原型制造、个性化制造等领域应用。这开启了新的制造方式。
   - 2010年,工业4.0概念的提出:德国提出了工业4.0概念。这标志着制造业开始向智能化、网络化方向发展。

%generated_Cutting_edge_field_
   \newpage
   \subsection*{推荐专业前沿领域}
   以下信息介绍了为推荐的3个专业分别在学术界和工业界中的较前沿的领域:
   1. 计算机科学与技术
   学术界:
   - 人工智能:人工智能是计算机科学的一个重要分支,它试图理解、设计和实现具有智能的系统。这包括但不限于自然语言处理、计算机视觉、机器学习等。随着深度学习的发展,人工智能已经在许多领域取得了显著的进步,如自动驾驶、医疗诊断等。
   - 量子计算:量子计算是一种基于量子力学原理进行信息处理和传输的新型计算方式。它利用量子比特(qubit)进行运算,相比传统二进制运算方式,其并行性和速度有着显著优势。目前,全球各大科研机构都在积极开展量子计算研究。
   - 数据科学:数据科学是一门交叉领域,涵盖了统计、数据挖掘、预测分析等多个方面。随着大数据时代的到来,如何从海量数据中提取有价值信息成为了一个重要问题。

   工业界:
   - 云计算:云计算通过网络提供按需使用的共享资源和服务。企业可以根据需要租用存储空间、计算能力等,大大降低了硬件投入和维护成本。
   - 物联网:物联网是通过网络将各种物理设备连接起来,实现信息的交换和通信。这包括智能家居、工业4.0等应用场景。
   - 区块链技术:区块链技术是一种分布式数据库技术,它可以确保数据的透明性、安全性和不可篡改性。目前已经被广泛应用于金融、供应链管理等领域。

2. 电子信息工程
   学术界:
   - 光电子学:光电子学研究光与电子的相互作用。这包括激光器、光纤通信、光探测器等设备的设计和制造。
   - 微纳电子学:微纳电子学主要研究微米或纳米尺度下的电子设备和系统。这涵盖了半导体材料、集成电路设计等多个方面。
   - 无线通信:无线通信是通过无线媒介进行信息传输的一种方式。随着5G时代的到来,如何提高传输速率、降低延迟成为了重要问题。

   工业界:
   - 5G通信:5G是第五代移动通信技术,它具有高速率、低延迟、大连接数等特点。目前,全球各大电信运营商都在积极部署5G网络。
   - 物联网:物联网通过网络将各种物理设备连接起来,实现信息的交换和通信。这包括智能家居、工业4.0等应用场景。
   - 自动驾驶:自动驾驶需要依赖多种传感器进行环境感知,并通过复杂的算法进行决策。其中,雷达、激光雷达和摄像头等电子设备的设计和制造是关键。

3. 机械工程
   学术界:
   - 先进制造技术:先进制造技术包括精密加工、微纳米制造、3D打印等新型制造方式。这些技术可以提高生产效率,降低成本,并且能够生产出传统方法无法实现的复杂结构。
   - 机器人学:机器人学研究如何设计和控制机器人。这包括了机械设计、控制理论、计算机视觉等多个方面。
   - 材料科学:材料科学研究新型材料的设计和制造。这包括了金属、陶瓷、塑料等多种类型的材料。

   工业界:
   - 智能制造:智能制造是通过引入信息技术和智能设备,实现生产过程的自动化和智能化。这包括了工业4.0、物联网等技术。
   - 新能源汽车:新能源汽车是指使用非传统燃料或者采用先进的动力系统的汽车。其中,电动汽车和混合动力汽车是最主要的两种类型。
   - 航空航天:航空航天工程涉及到飞机、火箭等设备的设计和制造。随着商业航天市场的发展,这个领域正在迎来新一轮的发展机遇。

%generated_Visualization_p1_
   \newpage
   \subsection*{学生学科兴趣和能力的多维度评估可视化}
   由于提供的信息不足,无法对学生在各个学科的五个维度进行打分。我们需要更多关于学生在这些学科中的表现、兴趣、实践应用能力、创新能力和对未来投入意愿的信息。例如,他们在课堂上的表现如何?他们是否对这些学科有兴趣?他们是否有将所学应用到实际问题中的经验?他们是否有创新思维?他们是否愿意在这些领域内投入更多时间和精力?

此外,从问卷结果来看,该学生似乎没有明确的专业偏好或者难以选择专业。这可能是因为他们从未考虑过专业选择问题。因此,在评估他们在各个学科上的五个维度时,也需要考虑到这一点。

总之,我们需要更多具体信息才能进行准确评估。
   % \input{figure_tex/radar_plot}
   % 释:上图展示了\username同学在我们推荐的三个最佳匹配专业和学科四大类里基于不同领域的五大关键特征的可视化分析。不同领域的五大关键特征分别为学生在该领域的知识掌握度,学生对该领域的热爱程度,学生对该领域学习内容的实用与应用能力,学生在该领域的创新能力,以及学生未来在该领域愿意投入的意愿。我们将可视化图片分成了五个维度,从一至五,每条彩色的线与数字的交叉点代表了学生在该专业或领域里的关键特征得分。分数越高,代表学生在学习该专业或领域时会与其五大特征匹配。
   
   
   
%generated_Visualization_p2_
   \subsection*{基于推荐专业的高中活动规划}
   根据您提供的信息,我们为同学提规划了基于3个不同专业发展方向的高中活动,可以供您参考:
   计算机科学与技术:
- 知识掌握程度:Jake在问卷中并未明确表示对计算机科学与技术的知识掌握程度,但他愿意尝试新事物,这可能有助于他在该领域的学习。因此,Jake在计算机科学与技术领域的知识掌握程度可能较一般,得分为3。
- 热爱程度:Jake并未明确表示对计算机科学与技术的热爱程度。然而,考虑到他愿意尝试新事物和对大学专业选择尚未确定,我们可以推测他可能对该专业有一定的兴趣。因此,Jake在计算机科学与技术领域的热爱程度可能较一般,得分为3。
- 实践实用能力:Jake没有提供关于他在实践应用能力方面的信息。但是,在当前社会环境下,计算机科学与技术是一个非常实用且广泛应用的领域。因此,如果Jake选择了这个专业,并投入到相关工作中去,则其实践实用能力有很大潜力得分为4。
- 创新能力:由于缺乏具体信息无法评估Jake在创新能力方面的表现,但是计算机科学与技术是一个需要不断创新和发展的领域,如果Jake愿意尝试新事物并投入到这个领域中去,他在创新能力方面有很大的提升空间。因此,Jake在计算机科学与技术领域的创新能力可能较一般,得分为3。
- 对未来的投入意愿:Jake对于大学专业选择尚未确定,并且他对于父母可提供资源的在意程度为3分,这可能表示他对于未来有一定的投入意愿。因此,Jake在计算机科学与技术领域对未来的投入意愿可能较高,得分为4。

电子信息工程:
- 知识掌握程度:由于缺乏具体信息无法评估Jake在电子信息工程方面的知识掌握程度。但是考虑到他愿意尝试新事物,并且对大学专业选择尚未确定,则其知识掌握程度可能较一般,得分为3。
- 热爱程度:同样由于缺乏具体信息无法评估Jake对电子信息工程专业的热爱程度。然而考虑到他愿意尝试新事物,并且对大学专业选择尚未确定,我们可以推测他可能对该专业有一定的兴趣。因此,Jake在电子信息工程领域的热爱程度可能较一般,得分为3。
- 实践实用能力:Jake没有提供关于他在实践应用能力方面的信息。但是,在当前社会环境下,电子信息工程是一个非常实用且广泛应用的领域。因此,如果Jake选择了这个专业,并投入到相关工作中去,则其实践实用能力有很大潜力得分为4。
- 创新能力:由于缺乏具体信息无法评估Jake在创新能力方面的表现,但是电子信息工程是一个需要不断创新和发展的领域,如果Jake愿意尝试新事物并投入到这个领域中去,他在创新能力方面有很大的提升空间。因此,Jake在电子信息工程领域的创新能力可能较一般,得分为3。
- 对未来的投入意愿:Jake对于大学专业选择尚未确定,并且他对于父母可提供资源的在意程度为3分,这可能表示他对于未来有一定的投入意愿。因此,Jake在电子信息工程领域对未来的投入意愿可能较高,得分为4。

机械工程:
- 知识掌握程度:由于缺乏具体信息无法评估Jake在机械工程方面的知识掌握程度。但是考虑到他愿意尝试新事物,并且对大学专业选择尚未确定,则其知识掌握程度可能较一般,得分为3。
- 热爱程度:同样由于缺乏具体信息无法评估Jake对机械工程专业的热爱程度。然而考虑到他愿意尝试新事物,并且对大学专业选择尚未确定,我们可以推测他可能对该专业有一定的兴趣。因此,Jake在机械工程领域的热爱程度可能较一般,得分为3。
- 实践实用能力:Jake没有提供关于他在实践应用能力方面的信息。但是,在当前社会环境下,机械工程是一个非常实用且广泛应用的领域。因此,如果Jake选择了这个专业,并投入到相关工作中去,则其实践实用能力有很大潜力得分为4。
- 创新能力:由于缺乏具体信息无法评估Jake在创新能力方面的表现,但是机械工程是一个需要不断创新和发展的领域,如果Jake愿意尝试新事物并投入到这个领域中去,他在创新能力方面有很大的提升空间。因此,Jake在机械工程领域的创新能力可能较一般,得分为3。
- 对未来的投入意愿:Jake对于大学专业选择尚未确定,并且他对于父母可提供资源的在意程度为3分,这可能表示他对于未来有一定的投入意愿。因此,Jake在机械工程领域对未来的投入意愿可能较高,得分为4。

综合考虑以上五个维度和Jake目前情况,在计算机科学与技术、电子信息工程和机械工程三个专业中都有一定的发展潜力。建议Jake根据自己实际情况和兴趣进一步了解这些专业,并做出最适合自己的选择。
   
   
   
 %generated_Highschool_activities_ 
   \newpage
   \hspace{0pt}
   \vspace{0cm}
   \subsection*{其他可选择专业}
   根据您提供的信息,我们为同学提供了其他较符合您要求的专业,可以供您参考:
   1. 计算机科学与技术(Computer Science and Technology):
一周以内:你可以组织一个校内的编程研讨会。邀请同学们一起参加,讨论最新的编程语言和技术,如Python、Java、人工智能等。你可以邀请计算机科学专业的教授或行业专家进行讲座,分享他们在实际工作中使用这些技术的经验和见解。通过这个活动,你将提升自己的编程知识和技能,并培养团队合作精神。

一个月以内:你可以参与在线平台上的计算机科学课程,如Coursera上的“Python for Everybody”或“Machine Learning”。这些课程将帮助你深入理解计算机科学理论,并提供实践项目来锻炼你的编程能力。同时,你也可以开始自己的小型研究项目,例如设计一个简单的网站或开发一个手机应用。

一年以内:在一年内,你可以寻找实习机会,在真实环境中应用所学知识。例如,在软件开发公司、互联网企业或大数据分析公司进行实习。此外,如果有条件,也可以尝试参加国际性比赛如Google Code Jam或ACM International Collegiate Programming Contest等。

背景提升规划:你可以考虑创建一个开源项目,如在GitHub上发布自己的代码。这将展示你的编程能力和解决实际问题的能力。同时,也可以参与其他开源项目,与全球的开发者一起合作,提升自己的团队协作和沟通能力。

2. 电子信息工程(Electronic Information Engineering):
一周以内:你可以组织一个电子制作研讨会。邀请同学们一起动手制作简单的电子设备,如LED灯、无线电等。通过这个活动,你将提高自己的动手能力和理解电路原理的能力。

一个月以内:你可以参加在线平台上关于电子信息工程的课程,如Coursera上的“Introduction to Electronics”。同时,也可以开始阅读相关书籍或论文,了解最新的研究成果和技术趋势。

一年以内:在一年内,你可以寻找实习机会,在真实环境中应用所学知识。例如,在通信公司、芯片设计公司或智能硬件公司进行实习。此外,也可以尝试参加国际性比赛如IEEE Student Paper Contest等。

背景提升规划:你可以考虑创建一个个人项目,例如设计并制作一个智能家居设备或机器人。这将展示你的电子设计和编程能力,同时也可以提升你的创新思维和解决问题的能力。

3. 机械工程(Mechanical Engineering):
一周以内:你可以组织一个机械制作研讨会。邀请同学们一起动手制作简单的机械设备,如小型风力发电机、模型汽车等。通过这个活动,你将提高自己的动手能力和理解机械原理的能力。

一个月以内:你可以参加在线平台上关于机械工程的课程,如Coursera上的“Introduction to Mechanical Engineering Design and Manufacturing”。同时,也可以开始阅读相关书籍或论文,了解最新的研究成果和技术趋势。

一年以内:在一年内,你可以寻找实习机会,在真实环境中应用所学知识。例如,在汽车制造公司、航空航天公司或重工业企业进行实习。此外,也可以尝试参加国际性比赛如SAE International Student Design Competition等。

背景提升规划:你可以考虑创建一个个人项目,例如设计并制作一个复杂的机械装置或系统。这将展示你的设计和制造能力,并提升你对整个产品开发过程的理解。


   \bigskip
   \bigskip
   \bigskip
   \noindent本报告基于大语言模型(LLM)收集和分析数据,旨在为学生未来的专业选择提供参考和启示。请注意,这些建议并非绝对,而是数据分析的结果。我们希望这些信息能在申请阶段为您提供帮助,但也请您审慎对待。

   \bigskip 
   \noindent感谢您的参与和支持!

   \noindent此致,

   \noindent\textbf{ChatIvy团队}

\printbibliography
\clearpage\end{CJK*}
\end{document}